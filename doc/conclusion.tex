In summary, the main contributions of this work is as follows. \textit{First}, a custom design and implementation of a smart grid plug load predictor and outlier identifier system. It is our belief that the custom design led to the significant throughput we observed with the given data set. In fact, the throughput of the system is much higher than what is reported because of the limitations of the broker process which is really not part of the predictor system. \textit{Second}, A scalable architecture to process dense smart plug load data streams. Every house process can be given it's own core to run on and so the scalability is limited to the number of house processes we want to process data for. \textit{Third}, a fast and lightweight C++ implementation of the median based prediction heuristic for predicting the load at smart plugs for different time windows. \textit{Fourth}, a method and code for efficiently sliding two different timeslice windows over a continuous stream of events.

The presented solution does not use work values.
We could use work values in case of Query 1 in order to compute total load for a given time window.
We would only use work values for the average load computation of longer time slices such as \{60 min, 120 min\}. This would lead to more precise computation of average load for a time slot of given time slice.
