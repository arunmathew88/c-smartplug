Smart Grid is quickly becoming the norm where there are many points of energy generation and the guiding principle is to consume the generated energy as close to the energy source as possible to avoid transmission over long distances.
In order to accurately channel energy from the source to the consumption points, it is necessary to rely on a system that can predict the energy requirements at different occupational units in real time, while consuming dense streams of millions of (energy consumption and load pattern) events per second.
The DEBS 2014 Grand Challenge\cite{ziekow2014challenge} focuses on this problem and specified two queries in this space.

While we investigated various open source complex event processing systems such as Esper\cite{Esper} and Padres\cite{Padres} as well as different execution environments and queuing products such as Erlang\cite{erlang} and ZeroMQ \cite{zeromq} socket, we chose instead to build this engine from ground up using C++ and BSD sockets. Fundamentally, we do not believe this is a CEP problem at it's core - it could be artificially conceived to be so but we have taken a different approach. In this paper, we present a simple and elegant solution with very high throughput rate (of over a million events per second) and scalability conceived purely for the purposes of answering the two queries given as the DEBS 2014 Grand Challenge.

The paper is structured as follows: In Section \ref{design}, we detail our design philosophy and approach. Sections \ref{query1} and \ref{query2} describe the system architecture and experimental results for Query 1 and 2 respectively.
We conclude the paper with the main contributions and future work in Section \ref{conclusions}.
