Smart grids are quickly becoming the norm where there are many points of energy generation and the guiding principle is to consume the generated energy as close to the energy source as possible to avoid transmission over long distances. In order to accurately channel energy from the source to the consumption points, it is necessary to rely on a system that can predict the energy requirements at different occupational units and identify outliers in real time, while consuming dense streams of millions of (energy consumption and load pattern) events per second. The DEBS 2014 Grand Challenge focuses on this problem and specified two queries in this space. Query 1 focused on a prediction strategy to accurately identify the possible load at various time scales in the future (1 min to 2 hours) at different occupational granularities (plug, house etc.) while Query 2 asked to identify outliers where the energy consumption patterns at a specific occupational unit has changed significantly over a short period of time. Both of these "queries" had to be answered by consuming a large and dense event stream consisting of hundreds of millions of load and work value events at different smart plugs in the smart grid. 

While we investigated various open source complex event processing systems such as Esper\cite{Esper} and Padres\cite{Padres} as well as different execution environments and queuing products such as Erlang\cite{erlang} and  ZeroMQ sockets \cite{zeromq} we chose instead to build this engine from the ground up using C++ and BSD sockets. Fundamentally, we do not believe this is a CEP problem at it's core - it could be artificially conceived to be so but we have taken a different approach. In this paper, we present a simple and elegant solution with very high throughput rate (of over a million events per second at a very low CPU utilization of under 20\%) and scalability. Our system therefore is a customized high speed, high throughput, low utilization system conceived purely for the purposes of answering the two queries given as the DEBS 2014 Grand Challenge. 

The paper is structured as follows: In Section \ref{design}, we detail our design philosophy and approach. Sections \ref{query1} and \ref{query2} describe the system architecture and experimental results for Query 1 and 2 respectively. We present some of the ideas, which we could not implement as part of the current solution, in Section \ref{future} as future work. We conclude the paper with the main contributions in Section \ref{conclusions}.
